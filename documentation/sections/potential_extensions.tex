\documentclass[../main.tex]{subfiles}

\begin{document}

\section{Erweiterungsmöglichkeiten}

\vspace{10pt}

Eine mögliche Erweiterung wären benutzerdefinierte Operatoren für andere Präzedenzen. Dazu müssten wir dem Lexer im Voraus mitteilen, dass eine gewisse Folge von Spezialzeichen einer bestimmten Operatoren-Klasse zugeordnet ist. \\

\vspace{10pt}

\underline{Beispiel:} \\

Der Operator \dq\&\dq soll ein ADDOPR sein. Nun können wir folgenden Token erstellen: \\ 

(ADDOPR, \dq\&\dq) \\

Der Parser sucht nun nach einer Funktion "custom\_addopr\_26", analog zu dem zuvor beschriebenen MULTOPR. \\ 

Ein Weg, dem Lexer benutzerdefinierte Operatoren mitzuteilen, wäre eine Compiler-Direktive am Anfang des Quelltextes in der Form \dq//custom addopr \&\dq. \\ 

Die Direktive wird aufgrund der Notation in Form eines Kommentars von IML-Compilern, welche dieses Feature nicht unterstützen, ignoriert. 

\end{document}